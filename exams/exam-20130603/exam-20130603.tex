% $iD$
% Author:   Daniel Bosk <daniel.bosk@miun.se>
\documentclass[a4paper,addpoints]{miunexam}
\usepackage[T1]{fontenc}
\usepackage[utf8]{inputenc}
\usepackage[english,swedish]{babel}
\usepackage[hyphens]{url}
\usepackage{hyperref}
\usepackage{color}
\usepackage{varioref,prettyref}
\usepackage{SIunits}
\usepackage{pdfpages}
\usepackage{csquotes}
\usepackage[style=alphabetic,natbib=true]{biblatex}
\addbibresource{literature.bib}
\addbibresource{rfc.bib}
\usepackage[today,nofancy]{svninfo}
\usepackage[varioref,prettyref]{miunmisc}

\svnInfo $Id$

\author{%
  Daniel Bosk\\
  \small\texttt{\href{mailto:daniel.bosk@miun.se}{daniel.bosk@miun.se}}\\
  \small Telefon: 060-14\,8709
}
\courseid{DT012G}
\course{Informationssäkerhet och riskanalys}
\date{2013-06-03}
\examtype{Tentamen}

%\printanswers

\begin{document}

\maketitle
\thispagestyle{foot}

\section*{Instruktioner}
\noindent
Läs uppgifterna noggrant innan du börjar att lösa dem.
Läs igenom samtliga uppgifter innan du börjar att lösa den första.
Planera din skrivning utifrån den tidsbegränsning som anges nedan.
Besvara endast frågan, skriv inte svar som ej är relaterad till frågan.

Tänk på att skriva med ett korrekt språk; grammatik och meningsbyggnad är 
viktigt.
Svaret ska tydligt framgå.
Dina svar ska visa att du förstått, tänk på att utforma dem för att visa just 
detta.

Frågorna är \emph{ej} sorterade efter svårighetsgrad.

Lärare finns tillgänglig via telefon under tentan.
Svar på frågor som läraren anser berör samtliga kommer att publiceras i kursens
diskussionsforum.
Notera att diskussionsforumet ej kommer att övervakas under denna tentamen, 
ställ därför inga frågor där.

\begin{description}
  \item[Skrivtid] 3 juni 2013 kl.\ 13:00 till kl.\ 19:00.
    (6 timmar, inkl.\ kortare paus).
	\item[Hjälpmedel] Kurslitteratur, egna anteckningar och referensmaterial på
		webben.
	\item[Antal uppgifter] \numquestions
	\item[Antal poäng] \numpoints
\end{description}


\subsection*{Betygssättning}
\noindent
Denna tentamen betygsätts med betygen A, B och F.
För slutbetygen E, D och C krävs inte denna tentamen, då räcker att du är 
godkänd på samtliga obligatoriska inlämningsuppgifter samt projekt.

Slutbetyget baserar sig på medelbetyget från de obligatoriska momenten.
Om du får ett B på denna tentamen ökas ditt slutbetyg med ett betygsteg.
På samma sätt om du får ett A ökas ditt slutbetyg med två betygsteg.

Preliminära betygsgränser för B är minst \unit{50}{\%} och för A krävs minst 
\unit{90}{\%}.


\clearpage
\section*{Uppgifter}
Nedan följer uppgifterna, glöm inte att de är givna utan inbördes ordning.

\begin{questions}
  \question\label{q:pwdpolicy}
  Universitetets lösenordspolicy kräver minst åtta tecken.
  Dessa tecken ska vara minst tre gemener, tre versaler och två siffror -- 
  dessutom måste dessa finnas bland de första åtta tecknen i lösenordet.
  Detta ger \( 26^3 26^3 10^2 = 26^6 10^2 \approx 2^{35} \) antal möjliga 
  lösenord.
  Lösenordet måste dessutom bytas var tredje månad, vilket i sin tur ger risken 
  för lösenordssystem där användaren baserar det nya lösenordet på det gamla.

  Den något enklare lösenordspolicyn minst åtta tecken med gemener, versaler 
  och siffror -- utan krav på något antal inom de olika kategorierna -- ger \( 
  (26 + 26 + 10)^8 = 62^8 \approx 2^{48} \) antal lösenord.
  Denna policy har inga krav på giltighetstid hos ett lösenord.

  Resultatet av detta är en reduktion av komplexiteten från \( 62^8 \) ned till 
  \( 26^6 10^2 \).
  Detta utgör en relativ minskning av komplexiteten med \( 1-\frac{26^6 
  10^2}{62^8} = 0.99986 \), alltså 99.99 procent.

  Oavsett vilken av ovan givna policyer som används får användarna (sannolikt) 
  svaga lösenord.

  \begin{parts}
    \part[4] Förklara hur dessa lösenordspolicyer kan angripas.

    \begin{solution}
      De kan naturligtvis angripas genom en ordlisteattack.
      Eftersom att den första policyn har den givna utformningen behöver 
      ordlistan inte innehålla några gissningsförslag som innehåller 
      specialtecken.
      (Den bör inte innehålla specialtecken inom de första åtta tecknen för att 
      öka hastigheten för gissningarna, det är ju onödigt att gissa på något 
      som ändå inte tillåts enligt policyn.)

      Ett annat angreppssätt är råstyrkemetoden.
      Här kan tas hänsyn till att lösenordet är minst åtta tecken, och 
      sannolikt inte är särskilt mycket längre; att antalet versaler, gemener 
      och siffror är av givna antal.
      Om vi har ett system som klarar av \( 10^6\approx 2^{20} \) gissningar 
      per sekund och lösenordsrymden är cirka \( 2^{35} \), då har vi gått 
      igenom hela lösenordsrymden på \( 2^{15} \) sekunder, tillika nio timmar.
      Antar vi det mycket långsammare systemet med \( 10^3\approx 2^{10} \), 
      tar det istället 388 dagar.

      Båda angreppsmetoderna kan förfinas ytterligare genom att anta att 
      användarna använder någon form av system för utformandet av sina 
      lösenord, detta antagande baserar sig på det påtvingade lösenordsbytet 
      var tredje månad.
      (Detta gäller naturligtvis enbart den första policyn, då den andra inte 
      har några krav på lösenordsbyten.)
      Ordlistemetoden kan utökas med tillägg av diverse prefix eller suffix på 
      varje ord i ordlistan.
      På liknande sätt kan råstyrkeattacken anpassas för att testa lösenord 
      i en bättre ordning och förhoppningsvis avsluta innan sökrymden är 
      uttömd.
    \end{solution}

    \part[4] Ge ett förslag på en riktigt bra lösenordspolicy.
    Förslag utan motivering ger noll poäng.

    \begin{solution}
      De problem som uppstår med ovan givna lösenordspolicyer är dels att 
      lösenordsrymden inskränks, speciellt av den första policyn.
      Det andra problemet är lösenordsbytet, vilket kan få användarna att 
      tillämpa system för lösenordsbytena; exempelvis att öka på en siffra 
      i slutet av lösenordet.
      Anledningen till detta är minnesbarhet.
      Användarna vill ha ett lösenord som är enkelt att komma ihåg, har de lärt 
      sig ett lösenord och sedan gör en liten modifiering då blir det enklare 
      att komma ihåg än att lära sig ett helt nytt lösenord.

      I en undersökning av \citet{Komanduri2011opa} visar det sig att den 
      lösenordspolicy som ger högst entropi är den som enbart kräver minst 16 
      tecken.
      Entropin används här för att beteckna den faktiska lösenordsrymden, många 
      lösenordskombinationer kommer aldrig att användas och entropin är ett 
      mått på just hur många som faktiskt kommer att användas.
      Denna lösenordspolicy ger också en av de högsta minnesbarheterna 
      i undersökningen.

      Om vi antar att användare kommer att välja en väldigt enkelt lösenord 
      enligt denna policy, exempelvis bara 16 gemener, då kommer sökrymden för 
      en råstyrkeattack att bli \( 26^{16}\approx 2^{75} \).
      Med vår snabbaste angripar ovan kommer det alltså att ta \( 10^{15} \) år 
      att gå igenom hela sökrymden.
      Notera att detta är ett minimum då medellängden för lösenorden enligt 
      denna policy var längre än 16 tecken \cite{Komanduri2011opa}.

      Låt oss anta att medellängden för ord i SAOL \cite{SAOL} är fyra tecken.
      För att kunna ha en svensk mening som lösenord måste vi då ha en mening 
      beståendes av minst fyra svenska ord.
      Om vi då använder SAOL för att angripa dessa lösenord, då har vi att 
      sökrymden är \( 125000^4\approx 2^{67} \) och det skulle fortfarande ta 
      den snabbaste angriparen ovan \( 10^{12} \) år att gå igenom alla 
      gissningarna.

      Utifrån detta kan vi konstatera att vi egentligen inte skulle behöva 
      någon livslängd för lösenord.
      Problemen som kan uppstå är om någon iaktar när lösenordet skrivs in 
      eller dylika problem.
      Därför kan det vara bra att införa en livslängd för lösenord på ett till 
      fem år, då blir det mindre frustration för användarna men ändå viss 
      förnyelse av lösenorden.
    \end{solution}
  \end{parts}

  \question\label{q:terminologi}
  Definiera följande begrepp:
  \begin{parts}
    \part[1] pålitlighet (trust),
    \part[1] pålitlig (trustworthy),
    \part[1] sekretess (secrecy),
    \part[1] konfidentialitet (confidentiality),
    \part[1] personlig integritet (privacy),
    \part[1] integritet (integrity), och
    \part[1] autenticitet (authenticity).
  \end{parts}

  \begin{solution}
    \citet[ss.\ 13--14]{Anderson2008sea} definierar begreppen enligt följande:
    \begin{description}
      \item[Pålitlighet] Ett system eller principal som innehar pålitlighet 
        (\foreignlanguage{english}{is trusted}) är ett system eller principal 
        som kan bryta din säkerhetspolicy.

      \item[Pålitlig] Ett system eller principal som är pålitlig 
        (\foreignlanguage{english}{is trustworthy}) är ett system eller 
        principal som inte kommer att misslyckas.
        (Den kommer alltså inte att bryta din säkerhetspolicy.)

        Ett exempel för att illustrera skillnaden ges av följande citat: 
        \begin{quote}
          \foreignlanguage{english}{
          ''if an NSA employee is observed in a toilet stall at Baltimore 
          Washington airport selling key material to a Chinese diplomat, then 
          (assuming his operation was not authorized) we can describe him as 
          `trusted but not trustworthy''' \cite[s.\  13]{Anderson2008sea}.
          }
        \end{quote}

      \item[Sekretess] Sekretess är en teknisk term för effekten av en mekanism 
        som begränsar antalet principals som kan ta del av information.

      \item[Konfidentialitet] Konfidentialitet syftar till att tillhandahålla 
        sekretess för andra principals hemliga information.

      \item[Personlig integritet] Detta är förmågan eller rätten att kunna 
        skydda sin personliga information.
        Det gäller alltså bara individer, exempelvis företag har ingen 
        personlig integritet.

      \item[Integritet] Detta är en teknisk term för egenskapen att data 
        förblir oförändrat, eller, om förändring sker ska den inte förbli 
        obemärkt.

      \item[Autenticitet] Detta begrepp innefattar integritet och fräshhet.
        Om kommunikation spelas in och sedan spelas upp vid ett annat 
        tillfälle, då kommer integriteten att ha bevarats men inte fräshheten 
        -- alltså är en återuppspelning inte autentisk.
    \end{description}
    Dessa definitioner stämmer även överens med RFC 4949 \cite{rfc4949}.
  \end{solution}

  \question\label{q:psykologi}
  Människans psykologi spelar en stor roll för säkerheten hos olika system.
  \begin{parts}
    \part[2] Förklara översiktligt varför psykologin är viktig inom 
    säkerhetsområdet.
    
    \begin{solution}
      Då systemen vi är beroende av och som ska upprätthålla vår säkerhet 
      handhas av människor blir psykologin genast viktig.
      Vi behöver psykologin inom säkerhetsområdet för att kunna ta hänsyn till 
      hur människor fungerar när vi konstruerar säkerhetssystem.
      Exempelvis, om vi gör ett system för komplext och användaren tycker att 
      komplexiteten är onödig, då kommer denne användare att aktivt försöka att 
      ta sig runt systemet -- kanske genom att skriva upp långa lösenord 
      istället för att lära sig dem utantill.

      Om vi däremot tar hänsyn till användarnas kognitiva begränsningar, då kan 
      vi konstruera system som både är säkra och enkla att använda.
    \end{solution}

    \part[4] Analysera en psykologibaserad attack och förklara varför den 
    fungerar.

    \begin{solution}
      En psykologibaserad attack utnyttjar svagheter hos användarna för att ta 
      sig runt ett säkerhetssystem, det är alltså inte säkerhetssystemen som 
      angrips.

      Ett exempel på en sådan attack kan vara att en användare får ett e-brev 
      som till synes är från banken och som innehåller en länk till en 
      inloggningssida, kallat nätfiske.
      Brevet kan be användaren att uppdatera någonting hos banken via internet.
      Ett förfarande beskrivs och sedan läggs till ''eller klicka på länken''.
      Med en förfarande som låter som att det kan ta fem till tio klick kommer 
      användaren sannolikt att välja enklicksalternativet.
      Notera att förfarandet måste vara korrekt för banken medan länken är till 
      en phishingsida.
      Utformandet kan leda till vad litteraturen \cite[s. 23]{Anderson2008sea} 
      kallar \emph{\foreignlanguage{english}{capture errors}}, att användaren 
      använder ett invant beteende: i detta fall att användaren klickar på 
      direktlänkar.

      Därutöver försöker nätfiskaren att få användaren att tillämpa fel regler 
      i situationen.
      Exempelvis, användaren kanske (omedvetet) lägger större vikt vid att ett 
      hänglås syns i webbläsaren för säker anslutning än att bankens namn är 
      rätt stavat i URL:en.
      Även att bankens namn finns med någonstans i URL:en kan vara en 
      tillräckligt stark regel för att användaren ska undvika att detektera den 
      felaktiga fiske-URL:en.
    \end{solution}
  \end{parts}

  \question[4]\label{q:mitm}
  Definiera begreppet mannen-i-mitten-attack 
  (\foreignlanguage{english}{man-in-the-middle attack}) och illustrera 
  begreppet med ett exempel.
  \begin{solution}
    Mannen-i-mitten-attack är enligt \citet[kap.\ 3]{Anderson2008sea} och 
    \citet{rfc4949} en aktiv avlyssning där angriparen fångar upp och 
    modifierar utvald information för att maskera sig som en eller flera av de 
    riktiga användarna.

    Ett exempel är en användare som ska logga in på banken.
    Användaren ansluter till angriparen som i sin tur ansluter till banken.
    Angriparen skickare vidare all data från banken till användaren.
    När användaren försöker att logga in skickar angriparen vidare 
    inloggningsinformationen till banken och angriparen är nu inloggad.
    Istället för att skicka den riktiga sidan för användaren denna gång kan 
    angriparen skicka en sidan som säger att inloggningen misslyckades.
    När användaren försöker logga in igen kan angriparen använda dessa 
    säkerhetskoder för att autentisera överföringar.
  \end{solution}

  \question\label{q:multilevel}
  Flernivåsäkerhet (\foreignlanguage{english}{multi-level security}) och 
  multilateral säkerhet (\foreignlanguage{english}{multi-lateral security}) är 
  två relaterade begrepp.

  \begin{parts}
    \part[2] Vad innebär flernivåsäkerhet?
    Förklara vad som ämnas med detta.

    \part[2] Vad innebär multilateral säkerhet?
    Förklara vad som ämnas med detta.

    \part[2] Beskriv fördelarna med att kombinera de båda metoderna.
  \end{parts}

  \begin{solution}
    Begreppet \emph{flernivåsäkerhet} innebär att information eller åtkomst 
    därutav värderas olika i en hierarkisk modell \cite[kap.\ 
    8]{Anderson2008sea}.
    Två välkända exempel är Bell--LaPadula-modellen (BLP) och Bibamodellen.
    BLP fokuserar på konfidentialitet och information värderas som mer eller 
    mindre konfidentiell.
    Den går ut på att en principal som har en given åtkomstbehörighet kan läsa 
    allt från sin egen nivå och alla nivåer nedan.
    Däremot kan samma användare enbart skriva på sin egen nivå eller på övre 
    nivåer.
    Detta är för att hindra informationsflöde från högre nivåer till lägre.

    Bibamodellen fungerar tvärt om då den fokuserar på att bibehålla 
    integriteten hos information istället för konfidentialiteten.

    \emph{Multilateral säkerhet}, å andra sidan, är inte hierarkisk.
    Den delar upp information horisontellt i olika avdelningar och det ska vara 
    omöjligt att information flödar från en avdelning till en annan.
    Välkända modeller inom multilateral säkerhet är Kinesiska muren och 
    Separation av uppgifter (\foreignlanguage{english}{Separation of duty}).

    Kinesiska muren innebär att det mellan varje avdelning skapas en ''mur'' 
    för att information inte ska kunna flöda emellan.
    Den innebär att om en anställd arbetar med en kund får den anställde inte 
    arbeta med kunder som är konkurrerande.

    Separation av uppgifter innebär exempelvis att det inte får vara samma 
    personer som driftar ett system som de som utvecklar det.
    Anledningen till detta är för att utvecklarna inte ska medvetet kunna bygga 
    in svagheter i systemet och sedan utnyttja dessa.
    Ett annat exempel är att en systemadministratör inte ska ha 
    skrivrättigheter till loggfilerna för ett system denne själv administrerar, 
    då denne i sådant fall skulle kunna göra vadhelst och sedan ta bort det 
    från loggarna.

    Fördelarna med att kombinera flernivå- och multilateral säkerhet är att ett 
    sådant system ger möjlighet till mer finskaliga säkerhetspolicyer.
    Exempelvis ska en läkare ha tillgång till patientjournaler, men enbart för 
    de patienter som denne ansvarar för.
    På samma sätt ska receptionisten kunna komma åt besökstider för alla 
    patienter (vid en given vårdinrättning), men inte journalerna.
  \end{solution}

  \question[4]\label{q:biometrics}
  Beskriv problemen som kan uppstå med biometriska system och förklara hur 
  dessa kan användas trots dessa begränsningar.

  \begin{solution}
    \citet{Anderson2008sea} anger några problem med biometriska system, 
    sammanfattningsvis är de
    \begin{itemize}
      \item att systemen fungerar dåligt eller inte alls för personer med 
        kroppskador, då systemen är anpassade för mycket begränsad naturlig 
        variation;
      \item att många system går att lura med hjälp av inspelningsattacker, 
        exempelvis att ett gammalt fingeravtryck visas upp för 
        fingeravtrycksläsaren eller att en inspelning spelas upp för ett 
        röstigenkänningsystem;
      \item att det är svårt för en mänsklig operatör att verifiera eller 
        falsifiera biometriska data;
      \item att precisionen i det biometriska systemet är för dålig för att 
        kunna skilja en användare från en annan.
    \end{itemize}

    En anledning till att de ändå kan användas är för att de kan ha en 
    avskräckande effekt, alltså en ren psykologisk effekt hos angripare (och 
    hos användare generellt).
    De kan även användas i kombination med en annan säkerhetsmekanism, 
    exempelvis ett lösenord, för att åstadkomma tvåfaktorautentisering.
  \end{solution}

\end{questions}

\printbibliography
\end{document}
