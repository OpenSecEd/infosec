% $Id$
% Author:  Daniel Bosk <daniel.bosk@miun.se>
\documentclass[svv,addpoints]{miunexam}
\usepackage[utf8]{inputenc}
\usepackage[T1]{fontenc}
\usepackage[swedish,english]{babel}
\usepackage[hyphens]{url}
\usepackage{hyperref}
\usepackage{color}
\usepackage{prettyref,varioref}
\usepackage{subfigure}
\usepackage{amsmath,amssymb}
\usepackage{listings}
\usepackage{authblk}

\usepackage{csquotes}
\MakeBlockQuote{<}{|}{>}
\EnableQuotes

\usepackage[natbib,style=alphabetic,maxbibnames=99,backend=bibtexu]{biblatex}
\addbibresource{literature.bib}

\usepackage[varioref,prettyref,listings]{miunmisc}

\printanswers

\examtype{Final exam}
\courseid{DV026G}
\course{Information Security}
\date{2015-06-05}
\author{%
  Daniel Bosk
}
\affil{%
  Department of Information and Communication Systems,\\
  Mid Sweden University, SE-851\,70 Sundsvall\\
  Email: \href{mailto:daniel.bosk@miun.se}{daniel.bosk@miun.se}\\
  Phone: 010-142\,8709
}

\DeclareMathOperator{\hmac}{HMAC}
\DeclareMathOperator{\xor}{\oplus}
\DeclareMathOperator{\concat}{||}

\begin{document}
\maketitle
\thispagestyle{foot}

\section*{Instructions}
\label{sec:Instructions}
Carefully read the questions before you start answering them.
Note the time limit of the exam and plan your answers accordingly.
Only answer the question, do not write about subjects remotely related to the
question.

Write your answers on separate sheets, not on the exam paper.
Only write on one side of the sheets.
Start each question on a new sheet.
Do not forget to \emph{motivate your answers.}

Make sure you write your answers clearly, if I cannot read an answer the answer
will be awarded no points---even if the answer is correct.
The questions are \emph{not} sorted by difficulty.

\begin{description}
  \item[Time] 5 hours.
  \item[Aids] Dictionary, course material and notes.
  \item[Maximum points] \numpoints
  \item[Questions] \numquestions
\end{description}

\subsection*{Preliminary grades}
The following grading criteria applies:
E \(\geq 50\%\),
D \(\geq 60\%\),
C \(\geq 70\%\), 
B \(\geq 80\%\),
A \(\geq 90\%\).
No question must be awarded zero points.


\clearpage
\section*{Questions}
The questions are given below.
They are not given in any particular order.

\begin{questions}
  \question\label{q:passwd:auth:E}
  Explain the following terms:
  \begin{parts}
    \part[1] Brute force
    \part[1] Dictionary attack
    \part[1] Hash table
    \part[1] Social engineering
    \part[1] Two-factor authentication
    \part[1] Phishing
  \end{parts}


  \question\label{q:passwd:infotheory:E:C:A}
  You are asked to estimate some password policies.
  The policies are the following:
  \begin{description}
    \item[comprehensive8]
      At least 8 characters consisting of upper and lower case, numbers and 
      special characters (assume the ones common with the numbers on 
      a keyboard).
    \item[randswedict3]
      Randomly choose three words from the Dictionary of the Swedish Language 
      (SAOL).
      This dictionary contains approximately 125\,000 words.
  \end{description}
  You should answer the following:
  \begin{parts}
    \part[4] Estimate the entropy for the passsword policies.
    (You may rely on the results in certain published research papers discussed 
    in the course for certain estimates.)
    \part[2] Decide how suitable they are for use in the home environment.
    \part[2] Decide how suitable they are for use in a web application.
  \end{parts}
  Note that you will not get any points without a motivation.


  \question[5]\label{q:msb:E:C}
  When you start in the company you realize that there is no Information 
  Security Management System (ISMS, Swe.~`ledningssystem för 
  informationssäkerhet').
  You go into your boss's office: persuade him and the management that you need 
  an ISMS.


  \question[5]\label{q:msb:E:C}
  Your boss is finally convinced that the company needs an Information Security 
  Management System (ISMS, Swe.~`ledningssystem för informationssäkerhet').
  He comes to ask you how an ISMS is best implemented, explain how that is 
  done.


  \question\label{q:foundations:E}
  Define the following terms:
  \begin{parts}
    \part[1] Trusted
    \part[1] Trustworthy
    \part[1] Secrecy
    \part[1] Confidentiality
    \part[1] Privacy
    \part[1] Integrity
    \part[1] Authenticity
  \end{parts}

  \begin{solution}
    \citet[ss.\ 13--14]{Anderson2008sea} definierar begreppen enligt följande:
    \begin{description}
      \item[Pålitlighet] Ett system eller principal som innehar pålitlighet 
        (\foreignlanguage{english}{is trusted}) är ett system eller principal 
        som kan bryta din säkerhetspolicy.

      \item[Pålitlig] Ett system eller principal som är pålitlig 
        (\foreignlanguage{english}{is trustworthy}) är ett system eller 
        principal som inte kommer att misslyckas.
        (Den kommer alltså inte att bryta din säkerhetspolicy.)

        Ett exempel för att illustrera skillnaden ges av följande citat: 
        \begin{quote}
          \foreignlanguage{english}{
            ''if an NSA employee is observed in a toilet stall at Baltimore 
            Washington airport selling key material to a Chinese diplomat, then 
            (assuming his operation was not authorized) we can describe him as 
            `trusted but not trustworthy''' \cite[s.\  13]{Anderson2008sea}.
          }
        \end{quote}

      \item[Sekretess] Sekretess är en teknisk term för effekten av en mekanism 
        som begränsar antalet principals som kan ta del av information.

      \item[Konfidentialitet] Konfidentialitet syftar till att tillhandahålla 
        sekretess för andra principals hemliga information.

      \item[Personlig integritet] Detta är förmågan eller rätten att kunna 
        skydda sin personliga information.
        Det gäller alltså bara individer, exempelvis företag har ingen 
        personlig integritet.

      \item[Integritet] Detta är en teknisk term för egenskapen att data 
        förblir oförändrat, eller, om förändring sker ska den inte förbli 
        obemärkt.

      \item[Autenticitet] Detta begrepp innefattar integritet och fräshhet.
        Om kommunikation spelas in och sedan spelas upp vid ett annat 
        tillfälle, då kommer integriteten att ha bevarats men inte fräshheten 
        -- alltså är en återuppspelning inte autentisk.
    \end{description}
    Dessa definitioner stämmer även överens med RFC 4949 \cite{rfc4949}.
  \end{solution}


  \question\label{q:usability:E:C:A}
  Human psychology is important in security.
  It is used in both security usability and social engineering.
  \begin{parts}
    \part[2] Give an overview of why psychology is important in 
    security.

    \begin{solution}
      Då systemen vi är beroende av och som ska upprätthålla vår säkerhet 
      handhas av människor blir psykologin genast viktig.
      Vi behöver psykologin inom säkerhetsområdet för att kunna ta hänsyn till 
      hur människor fungerar när vi konstruerar säkerhetssystem.
      Exempelvis, om vi gör ett system för komplext och användaren tycker att 
      komplexiteten är onödig, då kommer denne användare att aktivt försöka att 
      ta sig runt systemet -- kanske genom att skriva upp långa lösenord 
      istället för att lära sig dem utantill.

      Om vi däremot tar hänsyn till användarnas kognitiva begränsningar, då kan 
      vi konstruera system som både är säkra och enkla att använda.
    \end{solution}

    \part[4] Give an example of an attack which exploits weaknesses in human 
    psychology.
    Also explain why it works.

    \begin{solution}
      En psykologibaserad attack utnyttjar svagheter hos användarna för att ta 
      sig runt ett säkerhetssystem, det är alltså inte säkerhetssystemen som 
      angrips.

      Ett exempel på en sådan attack kan vara att en användare får ett e-brev 
      som till synes är från banken och som innehåller en länk till en 
      inloggningssida, kallat nätfiske.
      Brevet kan be användaren att uppdatera någonting hos banken via internet.
      Ett förfarande beskrivs och sedan läggs till ''eller klicka på länken''.
      Med en förfarande som låter som att det kan ta fem till tio klick kommer 
      användaren sannolikt att välja enklicksalternativet.
      Notera att förfarandet måste vara korrekt för banken medan länken är till 
      en phishingsida.
      Utformandet kan leda till vad litteraturen \cite[s. 23]{Anderson2008sea} 
      kallar \emph{\foreignlanguage{english}{capture errors}}, att användaren 
      använder ett invant beteende: i detta fall att användaren klickar på 
      direktlänkar.

      Därutöver försöker nätfiskaren att få användaren att tillämpa fel regler 
      i situationen.
      Exempelvis, användaren kanske (omedvetet) lägger större vikt vid att ett 
      hänglås syns i webbläsaren för säker anslutning än att bankens namn är 
      rätt stavat i URL:en.
      Även att bankens namn finns med någonstans i URL:en kan vara en 
      tillräckligt stark regel för att användaren ska undvika att detektera den 
      felaktiga fiske-URL:en.
    \end{solution}
  \end{parts}


  \question[5]\label{q:biometrics:E:C:A}
  Describe what problems can arise when using biometrical systems.
  Explain why they still can be used if employed properly.

  \begin{solution}
    \citet{Anderson2008sea} anger några problem med biometriska system, 
    sammanfattningsvis är de
    \begin{itemize}
      \item att systemen fungerar dåligt eller inte alls för personer med 
        kroppskador, då systemen är anpassade för mycket begränsad naturlig 
        variation;
      \item att många system går att lura med hjälp av inspelningsattacker, 
        exempelvis att ett gammalt fingeravtryck visas upp för 
        fingeravtrycksläsaren eller att en inspelning spelas upp för ett 
        röstigenkänningsystem;
      \item att det är svårt för en mänsklig operatör att verifiera eller 
        falsifiera biometriska data;
      \item att precisionen i det biometriska systemet är för dålig för att 
        kunna skilja en användare från en annan.
    \end{itemize}

    En anledning till att de ändå kan användas är för att de kan ha en 
    avskräckande effekt, alltså en ren psykologisk effekt hos angripare (och 
    hos användare generellt).
    De kan även användas i kombination med en annan säkerhetsmekanism, 
    exempelvis ett lösenord, för att åstadkomma tvåfaktorautentisering.
  \end{solution}
\end{questions}


\printbibliography
\end{document}
