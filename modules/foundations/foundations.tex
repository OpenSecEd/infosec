\emph{Summary:}
In this learning session we will cover the foundations of security.
By this we mean what security is all about, \eg what types of properties we are 
interested in and what we want to achieve in our security work.
We will also introduce the scientific method and particularly role in the area 
of security.

There are many human aspects to security, understanding them is important.
There are many ways to attack systems through their human operators.
We cover a variety of examples of such attacks and some aspects of human 
psychology.

\emph{Intended learning outcomes:}
After this session you should be able:
\begin{itemize}
  \item to \emph{understand} the what security is generally about.
  \item to \emph{differentiate} which types of scientific methods are 
    appropriate to answer a given question.
  \item to \emph{adopt} an adversarial thinking for situtions involving humans.
  \item to \emph{incorporate} basic psychology in the design of a system to 
    increase its security.
\end{itemize}

\emph{Reading:}
You should read Gollmann's chapter on \enquote{Foundations of Computer 
  Security}~\cite[Ch.~3]{Gollmann2011cs}.
There he attempts at a definition of Computer Security and related terms, \eg 
confidentiality, integrity, and availability, which we need for our treatment of 
the topic.
Anderson also covers this in Chapter 1 of~\cite{Anderson2008sea}.
He also treats a wider area than just \emph{computer} security, which is good 
for us, he covers many aspects of security in different examples.

The scientific method is covered in \enquote{How to Design Computer Security 
  Experiments}~\cite{HowToDesignSecurityExperiments}.
This paper discusses the scientific method of (parts of) the security field.
For a more in-depth reflection on the state of security as a scientific 
pursuit, we recommend \enquote{SoK: Science, Security and the Elusive Goal of 
  Security as a Scientific Pursuit}~\cite{SecurityAsAScience}.

Anderson gives a short summary of the psychology of users, their strengths and 
weaknesses, in Chapter 2 \enquote{Usability and Psychology} of 
\enquote{Security Engineering}~\cite{Anderson2008sea}.

\paragraph{References}
