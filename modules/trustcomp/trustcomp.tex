\emph{Summary:}
One can only do so much with software.
One problem with software and general purpose processors is that the software 
can be modified and the processor will still execute it.
Another is that, that running software cannot evaluate the processing 
environment which executes it.

Some examples: Alice had her laptop in her bag as it passed through the 
security check.
While she was busy with the scans, one customs official booted the laptop from 
a USB stick and installed a different boot loader.
Or, how can Alice even trust the computer when it is brand new?
Another aspect of this is to protect parts of the system from Alice herself, 
\eg this is what \ac{DRM} is all about.
We also have the compartmentalization of apps in a smartphone.
If Alice accidentally installs a malicious app, it shouldn't be able to 
compromize the banking app.
Here we will explore how to ensure the integrity of the computer system.

\emph{Intended learning outcomes:}
More concretely, after this session you should be able to
\begin{itemize}
  \item \emph{understand} the problem of trusted computing, its approaches to 
    solutions, the underlying assumptions and its limitations.
  \item \emph{analyse} different approaches to trusted computing and their 
    limitations and \emph{apply} them in a solution to a given problem.
\end{itemize}

\emph{Reading:}
We touch on the topics in Chapters
\begin{itemize}
  \item 4 (4.2.11--4.4.1),
  \item 16,
  \item 17,
  \item 18 (read until and including 18.2.1) and
  \item 23 (23.1--23.2)
\end{itemize}
in \citetitle{Anderson2008sea}~\cite{Anderson2008sea}.

For root-of-trust, there is the 
paper~\cite{EstablishRootOfTrustUnconditionally} by 
\citeauthor{EstablishRootOfTrustUnconditionally}.
Sections I and II are enough (we don't need more than an overview).
\Textcite{HDDmalware,USBmalware,BIOSmalware} provides some examples of 
real-world problems in this area.

For trusted execution-environments, we use Intel SGX as an example.
This is introduced by \textcite{IntelSGXTutorial}.
(For a very detailed exposition on SGX, see the work by 
\textcite{IntelSGXExplained}.)
