\section{Scope and aims}%
\label{sec:aim}
The aim of the course is that after the course you should be able to make 
high-level designs for secure solutions, i.e.\ combine relevant research results 
based on their high-level properties into a solution with the desired security, 
privacy and usability properties.
The problems and solutions can be in both the technical or organizational 
domain.

\subsection{\Aclp*{ILO}}
More concretely, after completing the course, you should be able to:
\begin{itemize}
  \item \emph{evaluate} the usability of security solutions and \emph{suggest} 
    improvements that improve usability and security.
  \item \emph{evaluate} threats, possible protection mechanisms and 
    \emph{design} a high-level approach to protection which considers 
    usability.
  \item \emph{overview} the field of information security, \emph{understand} 
    your own limits and where to search for solutions, \eg experts or published 
    research results that are relevant to the solution of a problem.
  \item \emph{analyse and apply} the results of published research in the 
    security field.
  \item \emph{apply} the Swedish Civil Contingency Agency's Framework for 
    Information Security Management Systems (ISO 27000) to \emph{analyse, assess 
      and improve} the information security in an organization.
\end{itemize}
The course has a variety of learning sessions designed to ensure that you learn 
these \acp{ILO}.
Each such session has a set of further specified \acp{ILO} that will help you 
achieve the \acp{ILO} above.

We will apply the following grading criteria.
% XXX Rewrite the grading criteria
\begin{description}
  \item[Grade E] You fulfil all the \acp{ILO} at the minimum level.
  \item[Grade A] You fulfil all the \acp{ILO},
    your evaluations and designs are extensive and well-founded in the research 
    literature, where applicable, and otherwise show deep insights in how to 
    use the building-blocks from the literature.
\end{description}
The grades B, C and D are intermediary grades.
When assessing the multi-dimensional domain of your knowledge we will try to 
make as fair a projection as possible onto this linear scale.



\subsection{Prerequisites}
\dots



\section{Course structure and overview}%
\label{CourseStructure}

The course is divided into three parts.
The first part of the course covers the foundations of security: what it is, 
how to evaluate new knowledge in the field.
This covers both purely technical aspects, but also includes human aspects such
as usability --- even if a system is proved secure, it will offer no security 
if its human users cannot use it.

The second part of the course covers information security on a strategic level, 
this concerns organizational management systems for information security: how 
to implement these and how to continuously run them in an organization.
It also includes threat and risk analysis.
The main material is produced by the Swedish Civil Contingencies Agency (MSB) 
and is based on the ISO 27000 standard.

The third part of the course covers the technical aspects: how to design 
security (and not to design security).
The focus in this part of the course is on security mechanisms and how to use 
these in secure systems.

\subsection{Teaching and tutoring}

The teaching of the course is oriented towards active learning.
I.e.\ the course consists of learning sessions which requires active 
participation.

Generally, you are expected to read the material in advance.
During the learning session the most important parts of the material will be 
discussed and you will perform some tasks to work with the topic in groups, 
i.e.\ to apply it to learn it more efficiently.
Some modules of the course will have several learning sessions linked together,
e.g.\ a starting seminar, followed by laboratory work which is then summarized 
and used in a final seminar.

\subsection{Schedule}

In \cref{Schedule} you will find an overview of the schedule for the course.
The detailed schedule can be found in the University's central scheduling 
system.
The details for each session can be found in \cref{CourseContents}.

\begin{table}
  % XXX Update time table
	\centering
  \caption{%
    A summary of the parts of the course and when they will be scheduled.
    The table is adapted to taking this course at half-time pace, i.e.\ 20 
    hours per week for 10 weeks.
  }\label{Schedule}
  \begin{tabular}{lp{9cm}}
    \toprule
    \textbf{Course week}	& \textbf{Work} \\
    \midrule
    1
      & Lecture: Course start/Foundations of security\\
      & Lecture: Security usability\\
      & Lecture: MSB's framework, part I\\
      & Start working on M1 (isms)\\
      & Lecture: MSB's framework, part II\\
      & Start working on M2, prepare S3 (risk)\\
    \midrule
    2
      & Lecture: Information theory\\
      & Lecture: Cryptography, part I\\
      & Lecture: Cryptography, part II\\
      & Lecture: Cryptography, part III\\
      & Grading: M1 (isms), M2 (risk)\\
    \midrule
    3
      & Lecture: Protocols, part I\\
      & Lecture: Protocols, part II\\
      & Lecture: Differential privacy, part I\\
      & Lecture: Differential privacy, part II\\
      & Seminar: S3 (risk)\\
    \midrule
    4
      & Lecture: Identification and authentication, part I\\
      & Lecture: Identification and authentication, part II\\
      & Lab: L4 (pwdeval)\\
      & Lab: L4 (pwdeval)\\
      & Seminar: L4 (pwdeval)\\
    \midrule
    5
      & Lecture: Access control, part I\\
      & Lecture: Access control, part II\\
      & Lab: L5 (pricomlab)\\
      & Lecture: Accountability\\
      & Lab: L5 (pricomlab)\\
    \midrule
    6
      & Seminar: L5 (pricomlab)\\
      & Lecture: Software security\\
      & Lecture: Trusted computing\\
      & Tutoring: P6 (devel)\\
      %& Tutoring: P6 (research)\\
    \midrule
    7
      & Tutoring: P6 (devel)\\
    \midrule
    8
      & Tutoring: P6 (devel)\\
    \midrule
    9
      & Tutoring: P6 (devel)\\
    \midrule
    10
      & Presentation: P6 (devel)\\
      & Second grading: M1 (isms), M2 (risk)\\
      & Second seminar: S3 (risk), L4 (pwdeval), L5 (pricomlab)\\
    \midrule
    +3 months
      & Second presentation: P6 (devel)\\
      & Final grading: M1 (isms), M2 (risk)\\
      & Final seminar: S3 (risk), L4 (pwdeval), L5 (pricomlab)\\
    \midrule
    +6 months
      & Final presentation: P6 (devel)\\
    \bottomrule
  \end{tabular}
\end{table}


\section{Course contents}%
\label{CourseContents}

This section summarizes each of the learning sessions, i.e.\ what they cover, 
what you are expected to learn and its reading material.

%\subsection{L0 Privacy is Dead}
%\input{privacydead-abstract.tex}
%
\subsection{Foundations of security}
\input{foundations-abstract.tex}

\subsection{Security usability}
\input{usability-abstract.tex}

\subsection{Managing information security}
\input{msbframework-abstract.tex}

\subsection{M1 Information security management system}
\input{ismsmemo-abstract.tex}

\subsection{M2 and S3 Assessment and risk analysis}
\input{risksem-abstract.tex}

\subsection{Information theory}
\input{infotheory-abstract.tex}

\subsection{Cryptography}
\input{crypto-abstract.tex}

\subsection{Protocols}
\input{proto-abstract.tex}

\subsection{Differential privacy}
\input{diffpriv-abstract.tex}

\subsection{Identification and authentication}
\input{auth-abstract.tex}

\subsection{L4 Evaluating authentication mechanisms}
\input{pwdeval-abstract.tex}

\subsection{Access control}
\input{accessctrl-abstract.tex}

\subsection{Accountability}
\input{accountability-abstract.tex}

\subsection{L5 Private communication}
\input{pricomlab-abstract.tex}

\subsection{Software security}
\input{software-abstract.tex}

\subsection{Trusted computing}
\input{trustcomp-abstract.tex}

%\subsection{P8 A short study in information security}
\subsection{P6 Integrating security and usability in development}
\input{project-abstract.tex}


\section{Assessment}%
\label{Assessment}
\mode*

This section explains how the course modules are graded and mapped to LADOK\@.
\Cref{LADOKTable} visualizes the relations between modules, credits, grades and 
LADOK\@.

\begin{frame}
\begin{table}
  \centering
  \setlength{\tabcolsep}{0.5em}
  \begin{tabular}{r r l l}
    \toprule
    LADOK & ECTS  & Grade       & Course assignments\\
    \midrule
    I101  & 1.0   & P, F        & M1, M2\\
    S101  & 1.0   & P, F        & S3\\
    L101  & 1.0   & P, F        & L4, L5\\
    R101  & 3.0   & A--F        & P6\\
    \midrule
    Total & 6.0   & A--F        & P6\\
    \bottomrule
  \end{tabular}
  \caption{%
    Table summarizing course modules and their mapping to LADOK\@.
    P means pass, F means fail.
    A--E are also passing grades, where A is the best.
  }\label{LADOKTable}
\end{table}
\end{frame}

The project report is graded from A to F, where A--E are for passing and F and 
Fx are for failing.
The project also includes an oral presentation which is graded pass (P) or fail 
(F), and is reported with the project to LADOK\@.
The grade of the project will also be the grade of the course total.



\subsection{Handed-in assignments}

In general, all hand-ins in the course must be in a \enquote{passable} 
condition; i.e.~they must be well-written, grammatically correct and without 
spelling errors, have citations and references according to~\cite{IEEEcitation} 
(see also~\cite{PurdueCitation} for a tutorial), and finally fulfil all 
requirements from the assignment instruction.
If you hand something in which is not in this condition, you will receive an 
F without further comment.

All material handed-in must be created by yourself, or, in the case of group 
assignments, created by you or one of the group members.
When you refer to or quote other texts, then you must provide a correct list of 
references and, in the case of quotations, the quoted text must be clearly 
marked as quoted.
If any part of the document is plagiarised you risk being suspended from study 
for a predetermined time, not exceeding six months, due to disciplinary 
offence.
If it is a group assignment, all group members will be held accountable for 
disciplinary offence unless it is clearly marked in the work who is responsible 
for the part containing the plagiarism.

If cooperation takes place without the assignment instruction explicitly 
allowing this, this will be regarded as a disciplinary offence with the risk of
being suspended for a predetermined time, not exceeding six months.
Unless otherwise stated, all assignments are to be done individually.

\subsection{\enquote{What if I'm not done in time?}}%
\label{sec:late}
\mode*

\mode<presentation>{%
  \begin{frame}
    \begin{itemize}
      \item You have three chances for grading per year.
      \item These are marked in the schedule.
      \item Thus there will be three deadlines per assignment until the next 
        time the course is given.
    \end{itemize}
  \end{frame}

  \begin{frame}
    \begin{itemize}
      \item No tutoring is planned after the course.
      \item If you want to ensure tutoring, it's during the course.
    \end{itemize}
  \end{frame}

  \begin{frame}
    \begin{remark}[If you predict you will not finish on time]
      \begin{itemize}
        \item Within three weeks of course start, deregister from the course.
        \item This allows you to reregister next time the course is given.

          \pause

        \item You must reregister to get access to the course the following 
          year.
        \item If you haven't cancelled, you'll be last in the queue.
      \end{itemize}
    \end{remark}
  \end{frame}
}

The deadlines on this course are of great importance, make sure to keep these!
%You must have completed the introductory assignment within its deadline.
%If you do not do this you will be deregistered from the course and your place 
%will be open to other students.

For seminars and presentations there will be three sessions during the course 
of a year, if you cannot make it to any of those you will have to return the 
next time the course is given; \ie up to a year later.
All of these sessions will be in the course schedule (in the Student Portal).
If you miss a deadline for the preparation for a seminar session, then you have 
to go for the next seminar even if the first seminar has not passed yet.

Written assignments are graded once during the course, most often shortly after 
the deadline of the assignment.
After the course you are offered two more attempts within a year.
In total you have three chances for having your assignments graded over the 
period of a year.
After that you should come back the next time the course is given.

No tutoring is planned after the end of the course, \ie after the last 
tutoring session scheduled in the course schedule.
If you are not done with your assignments during the course and want to be 
guaranteed tutoring you have to reregister for the next time the course is 
given.
Reregistration is a lower priority class of applicants for a course, all 
students applying for the course the first time have higher priority -- this 
includes reserves too.

%If you by the end of the course have a majority of the assignments left undone 
%you will have to reregister for the course the next time it is given.
%Whether you have completed the majority of the assignments or not is up to the 
%teacher to decide.
%Talk to the teacher to see if you have to reregister or can just hand in the 
%missing assignments.

Thus, if you feel that you will not be done with the course on time, it is 
better to stop the course at an early stage.
If you register a break within three weeks of the course start, you will be in 
the higher priority class of applicants the next time you apply for the course.
You can register such a break yourself in the Student Portal.




\printbibliography{}
