\mode*

\section{Scope and aims}%
\label{sec:aim}

\mode<presentation>{%
  \subsection{Scope}

  \begin{frame}
    \begin{itemize}

      \item The course treats a wide interpretation of Information Security.
      \item It treats both engineering and management.

        \pause

      \item The first part is about management.
      \item The second part is about engineering.

        \pause

      \item But the principles from the engineering parts can be applied in an 
        organization's process design too.

    \end{itemize}
  \end{frame}
}

The course treats information security from a user, organization and technical 
perspective.
The first part of the course concerns security on a strategic level, \ie 
managing security within an organization.
The second part of the course focuses on the operative parts, \ie security 
mechanisms and principles for design of secure systems.
In full, the course aims at giving you an understanding of threats to security 
and how to work to protect against these.

\mode<presentation>{%
  \subsection{Aims}
}

More concretely, the \acp{ILO} of the course are the following.
After completing the course, you should be able to:
\begin{frame}
\mode<presentation>{You should be able to}
\begin{itemize}

  \item \emph{apply} basic concepts and models in information security.

    \pause

  \item \emph{evaluate} the usability of security solutions and \emph{suggest} 
    improvements that improve usability and security.

  \item \emph{analyse} threats, possible protection mechanisms and \emph{design} 
    an approach to protection which considers usability.

    \pause

  \item \emph{apply} the Swedish Civil Contingency Agency's Framework for 
    Information Security Management Systems to \emph{analyse, assess and 
      improve} the information security in an organization.

    \pause

  \item \emph{review and apply} the results of published research in the 
    security field.

\end{itemize}
\end{frame}
The course has a variety of learning sessions designed to ensure that you learn 
these \acp{ILO}.
Each such session has a set of further specified \acp{ILO}:
\eg the first outcome above refers to \enquote{basic concepts and models}, 
the \acp{ILO} of a learning session would specify which concepts and models it 
covers.


\section{Course structure and content overview}%
\label{sec:outline}

The first part of the course covers information security on a strategic level, 
this concerns organizational management systems for information security: how 
to implement these and how to continuously run them in an organization.
The main material used for this part~\cite{%
  MSB2011itm,MSB2011sle,MSB2011p,%
	MSB2011v,MSB2011r,MSB2011gap,MSB2011gb,%
	MSB2011vs,MSB2011us,MSB2011upo,%
	MSB2011pg,MSB2011koa,MSB2011i,%
	MSB2011o,MSB2011g,MSB2011lg,%
	MSB2011ulo,MSB2011kf,MSB2011fa%
} is produced by the Swedish Civil Contingencies Agency (MSB) and is based on 
the ISO 27000 standard documents.

The second part of the course will focus on the content of Anderson's book 
\citetitle{Anderson2008sea}~\cite{Anderson2008sea} and Gollmann's book 
\citetitle{Gollmann2011cs}~\cite{Gollmann2011cs}.
The focus in the second part of the course is on security mechanisms and how to 
use these in secure systems.
There is also some additional material for this part of the course, \eg 
research papers and some other material.

\subsection{Teaching and tutoring}

\mode<presentation>{%
  \begin{frame}
    \begin{itemize}

      \item Teaching consists of several types of learning sessions.
      \item Most topics are covered only by lectures.
        
        \pause

      \item Some are complemented with seminars, hand-ins and labs.
      \item These are for combined learning and assessment.

        \pause

      \item These are focused to the first six weeks.
      \item The last four weeks are dedicated to the project.
      \item These weeks have weekly tutoring sessions.
        
    \end{itemize}
  \end{frame}
}

The course is taught using lectures, seminars, laboratory assignments and, 
finally, a project.
All assignments are numbered consecutively prefixed with an \enquote{L} for 
laboratory assignments, \enquote{S} for seminar assignments, \enquote{M} for 
memos and \enquote{P} for projects.

\subsection{Schedule}

You will find an outline for a schedule for the course in \cref{Schedule}.
You are free to follow this schedule or any schedule you make for yourself, but 
the learning and tutoring sessions, deadlines etc.\ will follow this schedule.
The detailed reading instructions for each item in the schedule can be found in 
the following sections.

\begin{frame}[allowframebreaks]
\begin{table}
	\centering
  \begin{tabular}{lp{9cm}}
    \toprule
    \textbf{Week}	& \textbf{Work} \\
    \midrule
    1
    & Lecture: Course start/Foundations of security\\
    & Lecture: Security usability\\
    %& Start working on L0 (privacy)\\
    \midrule
    2
    & Lecture: MSB's Framework, part I\\
    & Start working on M1 (isms)\\
    & Lecture: MSB's Framework, part II\\
    & Start working on M2, prepare S3 (risk)\\
    & Lecture: Records management\\
    \midrule
    3
    & Lecture: Information theory\\
    & Lecture: Cryptography, part I\\
    & Lecture: Cryptography, part II\\
    & First grading of M1 (isms), M2 (risk)\\
    \midrule
    4
    & Lecture: Identification and authentication, part I\\
    & Lecture: Identification and authentication, part II\\
    & Lecture: Protocols and formal verification\\
    & First seminar session S3 (risk)\\
    \midrule
    5
    & Lecture: Access control\\
    & Lecture: Accountability\\
    & Lab: L4 (pwdguess), L6 (pricomlab)\\
    & Seminar: S5 (pwdpolicies)\\
    \midrule
    6
    & Lecture: Trusted computing\\
    & Lecture: Side-channels\\
    & Lecture: Software security\\
    & Lab: L4 (pwdguess), L6 (pricomlab)\\
    \midrule
    7
    & Tutoring: P7 (research)\\
    & Lab: L4 (pwdguess), L6 (pricomlab)\\
    \midrule
    8
    & Tutoring: P7 (research)\\
    & Lab: L4 (pwdguess), L6 (pricomlab)\\
    %& Presentation for L0 (privacy)\\
    \midrule
    9
    & Tutoring: P7 (research)\\
    \midrule
    10
    & Presentation: P7 (research)\\
    & Second grading of M1 (isms), M2 (risk)\\
    & Seminar: second call for seminars (S3, S5)\\
    & Lab: final call for labs\\
    \midrule
    +3 months
    & Presentation: second call for presentations (P7)\\
    & Final grading of M1 (isms), M2 (risk)\\
    & Seminar: final call for seminars (S3, S5)\\
    \midrule
    +6 months
    & Presentation: final call for presentations (P7)\\
    \bottomrule
  \end{tabular}
  \caption{%
    A summary of the parts of the course and when they will (or should) be done.
    The table is adapted to taking this course at half-time pace, \ie 20 hours 
    per week for 10 weeks.
  }\label{Schedule}
\end{table}


\end{frame}


\section{Course content}

This section summarizes the material covered by the lectures and assignments, 
\ie what you should read for each of them.
It is divided by topics and ordered according to progression of the course.

\mode<article>{%
\subsection{Foundations of security}
\input{foundations-lit.tex}
}

\mode<article>{%
\subsection{Security usability}
\input{usability-lit.tex}
}

%\subsection{L0 Privacy is Dead}
%\input{privacy-lit.tex}

\mode<article>{%
\subsection{MSB's framework, part I}
\input{msbintro-lit.tex}
}

\mode<article>{%
\subsection{M1 Information security management system}
\input{ismsmemo-lit.tex}
}

\mode<article>{%
\subsection{MSB's framework, part II}
\input{msbforts-lit.tex}
}

\mode<article>{%
\subsection{M2 and S3 Assessment and risk analysis}
\input{risksem-lit.tex}
}

\mode<article>{%
\subsection{Information security from a records management perspective}
\input{recmgmt-abstract.tex}
}

\mode<article>{%
\subsection{Information theory}
\input{infotheory-lit.tex}
}

\mode<article>{%
\subsection{Cryptography}
\input{crypto-lit.tex}
}

\mode<article>{%
\subsection{Identification and authentication}
\input{auth-lit.tex}
}

\mode<article>{%
\subsection{L4 Password cracking and social engineering}
\input{pwdguess-lit.tex}
}

\mode<article>{%
\subsection{S5 Password policies}
\input{pwdpolicies-lit.tex}
}

\mode<article>{%
\subsection{Protocols and formal verification}
\input{fverif-lit.tex}
}

\mode<article>{%
\subsection{Access control}
\input{accessctrl-lit.tex}
}

%\subsection{Multi-Level and Multi-Lateral Security}
%\input{lvlltrl-lit.tex}

\mode<article>{%
\subsection{Accountability}
\input{accountability-lit.tex}
}

\mode<article>{%
\subsection{L6 Private communication}
\input{pricomlab-lit.tex}
}

\mode<article>{%
\subsection{Trusted computing}
\input{trustcomp-lit.tex}
}

\mode<article>{%
\subsection{Software security}
\input{software-lit.tex}
}

\mode<article>{%
\subsection{Course conclusion}
}

During this lecture we will shortly review the course and try to fit things into 
a bigger picture.

\subsection{P7 A short study in information security}

\mode<presentation>{%
  \begin{frame}
    \begin{itemize}
      \item Small independent study in information security.
      \item Aim is to practice your knowledge from the course.
      \item As well as deepen your knowledge in some parts.

        \pause

      \item And to assess that you reach the \acp{ILO} above.
    \end{itemize}
  \end{frame}

  \begin{frame}
    \begin{itemize}
      \item You are quite free in choosing.
      \item But the project must be connected to research.
      \item You must select and read relevant research papers.
    \end{itemize}
  \end{frame}
}

The project is a smaller study within the area of information security.
The idea is to deepen your knowledge in some areas of information security, so 
during the project you must select and read papers which are related to the 
course.
For example, you can focus on areas such as:
\begin{itemize}
  \item usable security,
  \item privacy enhancing technologies (PETs), or
  \item more advanced methods for guessing passwords.
\end{itemize}
These are just examples, you are free to choose the area and papers in 
corroboration with the tutor.
More details are provided in the separate instruction.


\section{Assessment}%
\label{Assessment}

\mode<presentation>{%
  \subsection{LADOK modules}
}

This section explains how the course modules are graded and mapped to LADOK\@.
\Cref{LADOKTable} visualizes the relations between modules, credits, grades and 
LADOK\@.

\begin{frame}
\begin{table}
  \centering
  \setlength{\tabcolsep}{0.5em}
  \begin{tabular}{r r l l}
    \toprule
    LADOK & Credits (ECTS)  & Grade       & Course Assignments\\
    \midrule
    I104  & 1.5             & P, F        & M1, M2, S3, S5\\
    L104  & 1.5             & P, F        & L4, L6\\
    R104  & 4.5             & A--F        & P7\\
    \midrule
    Total & 7.5             & A--F        & P7\\
    \bottomrule
  \end{tabular}
  \caption{%
    Table summarizing course modules and their mapping to LADOK\@.
    P means pass, F means fail.
    A--E are also passing grades, where A is the best.
  }\label{LADOKTable}
\end{table}
\end{frame}

The project is graded from A to F, where A--E are for passing and F and Fx are 
for failing.
The grade of the project will also be the grade of the course total.

\subsection{Handed-in assignments}

\mode<presentation>{%
  \begin{frame}
    \begin{itemize}
      \item Must be in \enquote{passable} condition.
      \item Otherwise rejection without comment.

        \pause

      \item No plagiarism accepted.
      \item When working in group: everyone accountable.
    \end{itemize}
  \end{frame}
}

In general, all hand-ins in the course must be in a \enquote{passable} 
condition; \ie they must be well-written, grammatically correct and without 
spelling errors, have citations and references according to~\cite{IEEEcitation} 
(see also~\cite{PurdueCitation} for a tutorial), and finally fulfil all 
requirements from the assignment instruction.
If you hand something in which is not in this condition, you will receive an 
F without further comment.

All material handed-in must be created by yourself, or, in the case of group 
assignments, created by you or one of the group members.
When you refer to or quote other texts, then you must provide a correct list of 
references and, in the case of quotations, the quoted text must be clearly 
marked as quoted.
If any part of the document is plagiarized you risk being suspended from study 
for a predetermined time, not exceeding six months, due to disciplinary 
offence.
If it is a group assignment, all group members will be held accountable for 
disciplinary offence unless it is clearly marked in the work who is responsible 
for the part containing the plagiarism.

If cooperation takes place without the assignment instruction explicitly 
allowing this, this will be regarded as a disciplinary offence with the risk of
being suspended for a predetermined time, not exceeding six months.
Unless otherwise stated, all assignments are to be done individually.

\subsection{\enquote{What if I'm not done in time?}}%
\label{sec:late}

\mode<presentation>{%
  \begin{frame}
    \begin{itemize}
      \item You have three chances for grading per year.
      \item These are marked in the schedule.
      \item Thus there will be three deadlines per assignment until the next 
        time the course is given.
    \end{itemize}
  \end{frame}

  \begin{frame}
    \begin{itemize}
      \item No tutoring is planned after the course.
      \item If you want to ensure tutoring, it's during the course.
    \end{itemize}
  \end{frame}

  \begin{frame}
    \begin{alertblock}{If you predict you will not finish on time}
      \begin{itemize}
        \item Within three weeks of course start, deregister from the course.
        \item This allows you to reregister next time the course is given.

          \pause

        \item You must reregister to get access to the course the following 
          year.
        \item If you haven't cancelled, you'll be last in the queue.
      \end{itemize}
    \end{alertblock}
  \end{frame}
}

The deadlines on this course are of great importance, make sure to keep these!
%You must have completed the introductory assignment within its deadline.
%If you do not do this you will be deregistered from the course and your place 
%will be open to other students.

For seminars and presentations there will be three sessions during the course 
of a year, if you cannot make it to any of those you will have to return the 
next time the course is given; \ie up to a year later.
All of these sessions will be in the course schedule (in the Student Portal).
If you miss a deadline for the preparation for a seminar session, then you have 
to go for the next seminar even if the first seminar has not passed yet.

Written assignments are graded once during the course, most often shortly after 
the deadline of the assignment.
After the course you are offered two more attempts within a year.
In total you have three chances for having your assignments graded over the 
period of a year.
After that you should come back the next time the course is given.

No tutoring is planned after the end of the course, \ie after the last 
tutoring session scheduled in the course schedule.
If you are not done with your assignments during the course and want to be 
guaranteed tutoring you have to reregister for the next time the course is 
given.
Reregistration is a lower priority class of applicants for a course, all 
students applying for the course the first time have higher priority -- this 
includes reserves too.

%If you by the end of the course have a majority of the assignments left undone 
%you will have to reregister for the course the next time it is given.
%Whether you have completed the majority of the assignments or not is up to the 
%teacher to decide.
%Talk to the teacher to see if you have to reregister or can just hand in the 
%missing assignments.

Thus, if you feel that you will not be done with the course on time, it is 
better to stop the course at an early stage.
If you register a break within three weeks of the course start, you will be in 
the higher priority class of applicants the next time you apply for the course.
You can register such a break yourself in the Student Portal.


\printbibliography{}
