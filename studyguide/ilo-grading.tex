More concretely, after completing the course, you should be able to:
\begin{itemize}
  \item \emph{evaluate} the usability of security solutions and \emph{suggest} 
    improvements that improve usability and security.
  \item \emph{evaluate} threats, possible protection mechanisms and 
    \emph{design} a high-level approach to protection which considers 
    usability.
  \item \emph{overview} the field of information security, \emph{understand} 
    your own limits and where to search for solutions, \eg experts or published 
    research results that are relevant to the solution of a problem.
  \item \emph{analyse and apply} the results of published research in the 
    security field.
  \item \emph{apply} the Swedish Civil Contingency Agency's Framework for 
    Information Security Management Systems (ISO 27000) to \emph{analyse, assess 
      and improve} the information security in an organization.
\end{itemize}
The course has a variety of learning sessions designed to ensure that you learn 
these \acp{ILO}.
Each such session has a set of further specified \acp{ILO} that will help you 
achieve the \acp{ILO} above.

We will apply the following grading criteria.
% XXX Rewrite the grading criteria
\begin{description}
  \item[Grade E] You fulfil all the \acp{ILO} at the minimum level.
  \item[Grade A] You fulfil all the \acp{ILO},
    your evaluations and designs are extensive and well-founded in the research 
    literature, where applicable, and otherwise show deep insights in how to 
    use the building-blocks from the literature.
\end{description}
The grades B, C and D are intermediary grades.
When assessing the multi-dimensional domain of your knowledge we will try to 
make as fair a projection as possible onto this linear scale.

