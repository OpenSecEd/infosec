\mode<presentation>{%
  \begin{frame}
    \begin{itemize}
      \item You have three chances for grading per year.
      \item These are marked in the schedule.
      \item Thus there will be three deadlines per assignment until the next 
        time the course is given.
    \end{itemize}
  \end{frame}

  \begin{frame}
    \begin{itemize}
      \item No tutoring is planned after the course.
      \item If you want to ensure tutoring, it's during the course.
    \end{itemize}
  \end{frame}

  \begin{frame}
    \begin{alertblock}{If you predict you will not finish on time}
      \begin{itemize}
        \item Within three weeks of course start, deregister from the course.
        \item This allows you to reregister next time the course is given.

          \pause

        \item You must reregister to get access to the course the following 
          year.
        \item If you haven't cancelled, you'll be last in the queue.
      \end{itemize}
    \end{alertblock}
  \end{frame}
}

The deadlines on this course are of great importance, make sure to keep these!
%You must have completed the introductory assignment within its deadline.
%If you do not do this you will be deregistered from the course and your place 
%will be open to other students.

For seminars and presentations there will be three sessions during the course 
of a year, if you cannot make it to any of those you will have to return the 
next time the course is given; \ie up to a year later.
All of these sessions will be in the course schedule (in the Student Portal).
If you miss a deadline for the preparation for a seminar session, then you have 
to go for the next seminar even if the first seminar has not passed yet.

Written assignments are graded once during the course, most often shortly after 
the deadline of the assignment.
After the course you are offered two more attempts within a year.
In total you have three chances for having your assignments graded over the 
period of a year.
After that you should come back the next time the course is given.

No tutoring is planned after the end of the course, \ie after the last 
tutoring session scheduled in the course schedule.
If you are not done with your assignments during the course and want to be 
guaranteed tutoring you have to reregister for the next time the course is 
given.
Reregistration is a lower priority class of applicants for a course, all 
students applying for the course the first time have higher priority -- this 
includes reserves too.

%If you by the end of the course have a majority of the assignments left undone 
%you will have to reregister for the course the next time it is given.
%Whether you have completed the majority of the assignments or not is up to the 
%teacher to decide.
%Talk to the teacher to see if you have to reregister or can just hand in the 
%missing assignments.

Thus, if you feel that you will not be done with the course on time, it is 
better to stop the course at an early stage.
If you register a break within three weeks of the course start, you will be in 
the higher priority class of applicants the next time you apply for the course.
You can register such a break yourself in the Student Portal.

